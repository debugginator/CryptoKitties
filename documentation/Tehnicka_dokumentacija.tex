\documentclass[times, utf8, tehnicka_dokumentacija]{fer}
\usepackage{booktabs}
\usepackage{fancyhdr}
\usepackage{lastpage}
\usepackage{indentfirst}
\usepackage{url}
\usepackage{listings}
\usepackage{color}

\definecolor{dkgreen}{rgb}{0,0.6,0}
\definecolor{gray}{rgb}{0.5,0.5,0.5}
\definecolor{mauve}{rgb}{0.58,0,0.82}

\lstset{frame=tb,
  aboveskip=3mm,
  belowskip=3mm,
  showstringspaces=false,
  columns=flexible,
  basicstyle={\small\ttfamily},
  numbers=none,
  numberstyle=\tiny\color{gray},
  keywordstyle=\color{blue},
  morekeywords={function,internal,returns,uint256,uint32,uint16,address,emit,unit,return,is,contract},
  commentstyle=\color{dkgreen},
  stringstyle=\color{mauve},
  breaklines=true,
  breakatwhitespace=true,
  tabsize=4
}

\newarray\tim

\def \naslov {CryptoKitties na Ethereum mreži}
\def \ver {1.0}
\def \broj {4}
\readarray{tim}{Ivan Mihaljević&Blaž Bagić&Karlo Vrbić&Mateo Gobin}
\def \mentor {Federico Matteo Benčić}

\fancypagestyle{plain}

\begin{document}

\title{\naslov}
\verzija{\ver}
\author{\tim}
\size{\broj}
\voditelj{\mentor}

\maketitle
\tableofcontents

\chapter{Uvod}
CryptoKitties je sustav za implementirati sustav uzgajanje i preprodaju virtualnih mačaka. U CryptoKitties sustavu,
korisnici prikupljaju i uzgajaju virtualne mačke! Svaka mačka ima jedinstveni genom koji ju definira. Korisnici mogu
uzgajati svoje mačke kako bi stvorili nove mačke koje se raylikuju od ostalih po svom genomu.

CryptoKitties je implementiran kao raspodijeljeni sustav zasnovan na {\it blockhain} tehnologiji. {\it Blockhain} je
tehnologija pomoću koje implementirati raspodijeljene i decentralizirane sustave, a funkcionira kao lista zapisa, koji
nazivamo {\it blokovima}, a koji su povezani jedni s drugima tako tvoreći jedan lanac blokova. Iako CryptoKitties nije
digitalna valuta, nudi istu sigurnost: svaka je mačka jedinstvena i zna se točno njen vlasnik. Ne može se ponoviti,
oduzeti ili uništiti.

\chapter{Tehnologija}
Sustav je implementiran korištenjem raznih tehnologija i knjižnica kao npr. {\it Ethereum}, {\it Truffle},
{\it JavaScript} i sl.

\section{Ethereum}
{\it Ethereum}  je decentralizirana platforma koja ima mogućnost pokretanja tzv. {\it smart contract-a}, tj. aplikacija
koje rade točno onako kako su programirane bez ikakve mogućnosti zastoja, cenzure, prijevare ili uplitanja trećih
strana. Na taj način možemo grantirati sigurnost pri preprodaji i uzgajanju mačaka.

\section{Truffle}
Za razvoj i testiranje pametnih ugovora korišten je alat {\it Truffle} koji omogućava simuliranje {\it Ethereum} mreže
lokalno na osobnom računalu. Ovaj alat nam omogućava kompilaciju pametnih ugovora, automatsko testiranje pomoću
{\it JavaScript} knjižnica {\it Mocha} i {\it Chai}, skriptiran deployment i migracije i još mnogo funkcionalnosti koje
olakšavaju razvoj pametnih ugovora na {\it Ethereum} mreži.

\section{JavaScript}
{\it JavaScript} je skriptni programski jezik koji CryptoKitties sustav koristi za migracije i testiranje pametnih
ugovora. Za testiranje funkcionalnosti poput kreiranja, uzgajanja, prodaje ali i otkayivanje prodaje mačaka korištena
je posebna {\it JavaScript} knjižnica imena {\it Mocha} i {\it Chai}. Konkretno {\it Mocha} je knjižnica koja nam
omogućava asinkrono testiranje dok je knjižnica {\it Chai} tzv. {\it assertion library}, tj. pruža nam metode za
provjeru ispravnosti određenih vrijednosti.

\section{OpenZeppelin}
{\it OpenZeppelin} je knjižnica za siguran razvoj pametnih ugovora. Ona osigurava implementaciju standarda kao što su
ERC20 i ERC721, kao i komponente {\it Solidity} za izradu prilagođenih ugovora i složenijih decentraliziranih sustava.

\chapter{Tehničke značajke}

\section{Kreiranje mačaka}
Korisnicima se za početak mora pounditi određeni broj mačaka. U isječku koda \ref{lst:createkitty} vidimo kako se to
odvija. Metoda \lstinline|_createKitty| je odgovorna za stvaranje mačke i emitiranje događaja \lstinline|Birth|
definiranog na način prikazan u odsječku.

\begin{figure}
\begin{lstlisting}
event Birth(address owner, uint256 geneticCode, uint32 _parent1_id, uint32 _parent2_id, uint32 _id);

function _createKitty(
  address _owner,
  uint256 _geneticCode,
  uint32 _parent1_id,
  uint32 _parent2_id,
  uint16 _generation)
  internal
  returns (uint)
{
  Kitty memory _kitty = Kitty({
    geneticCode: _geneticCode,
    parent1_id: _parent1_id,
    parent2_id: _parent2_id,
    generation: _generation,
    birthTime: uint64(now)
  });
  uint32 _kittyId = uint32(kitties.push(_kitty) - 1);
  emit Birth(_owner, _geneticCode, _parent1_id, _parent2_id, _kittyId);
  return _kittyId;
}
\end{lstlisting}
\caption{Kod za kreiranje mačke}
\label{lst:createkitty}
\end{figure}

\section{Prodaja mačaka}
Jedna od glavnih značajki ovog sustava je prodaja mačaka. Korisnici između sebe mogu prodavati mačke koje posjeduju za
određenu cijenu. To podrazumijeva kupovanje mačaka sa tržišta, stavljanje mačke na tržište i povlačenje mačke sa tržišta
ukoliko se korisnik predomislio. Pametni ugovor za tržište mačaka je prikazan na \ref{lst:kittymarket}

\begin{figure}
\begin{lstlisting}
contract KittyMarket is KittyOwnership {
  mapping (uint32 => uint256) public kittyIndexToPrice;

  function putKittyUpForSale(uint32 kittyId, uint256 price) external {
    require(ownerOf(kittyId) == msg.sender);
    kittyIndexToPrice[kittyId] = price;
  }

  function cancelSale(uint32 kittyId) external {
    require(ownerOf(kittyId) == msg.sender);
    kittyIndexToPrice[kittyId] = 0;
  }

  function buyKitty(uint32 kittyId) external payable {
    uint256 price = kittyIndexToPrice[kittyId];
    address payable owner = address(uint160(ownerOf(kittyId)));

    require(owner != msg.sender);
    require(price > 0);
    require(msg.value >= price);

    kittyIndexToPrice[kittyId] = 0;
    _transferFrom(ownerOf(kittyId), msg.sender, kittyId);
    owner.transfer(msg.value);
  }
}
\end{lstlisting}
\caption{Pametni ugovor za tržište mačaka}
\label{lst:kittymarket}
\end{figure}

\section{Uzgajanje mačaka}
Uzgajanje mačaka je značajka koja korisnicima pruža način da parenjem dviju mačaka dobiju dijete mačku. Kako mačke u
sustavu nemaju spol nije bitno koje dvije mačke se pare. Pametni ugovor za parenje mačaka prikazan je na
\ref{lst:kittybreeding} isječku koda.

\begin{figure}
\begin{lstlisting}
contract KittyBreeding is KittyOwnership {

  function breedWith(uint32 parent1Id, uint32 parent2Id)
    public
    returns(uint)
  {
    require(parent1Id >= 0 && parent2Id >= 0);
    require(parent1Id != parent2Id);
    require(ownerOf(parent1Id) == msg.sender);
    require(ownerOf(parent2Id) == msg.sender);

    Kitty storage parent1 = kitties[parent1Id];
    Kitty storage parent2 = kitties[parent2Id];

    require(parent1.birthTime != 0 && parent2.birthTime != 0);

    uint16 generation = uint16(max(parent1.generation, parent2.generation) + 1);
    uint256 geneticCode = cantorPairingFunction(parent1.geneticCode, parent2.geneticCode);
    uint kittyId = _createKitty(msg.sender, geneticCode, parent1Id, parent2Id, generation);

    mint(msg.sender, uint256(kittyId));
    return kittyId;
  }

  function cantorPairingFunction(uint a, uint b) private pure returns (uint) {
    return (a+b)*(a+b+1)*b/2;
  }

  function max(uint a, uint b) private pure returns (uint) {
      return a > b ? a : b;
  }
}
\end{lstlisting}
\caption{Pametni ugovor za uzgajanje mačaka}
\label{lst:kittymarket}
\end{figure}

\chapter{Podjela rada}
Na razvoju sustava ponajviše je radio Blaž Bagić, dok je Ivan Mihaljević bio zadužen za pisanje testova. Dokumentacija su napisali Karlo Vrbić i Mateo Gobin.

\chapter{Zaključak}
Na implementaciji ovog sustava korištene su tehnologije za razvoj decentraliziranih raspodijeljenih sustava. Tehnlogije poput {\it blockchaina} i {\it NFT}-a i svakako će biti zanimljivo pratiti kako će ove tehnologije utjecati i mjenjati način razvoja informatičkih sustava.

Izrada {\it CryptoKitties} je bila zabavno i poučno isksutvo kroz koje smo naučili osnove {\it Ethereum} pametnih ugovora i {\it Non-Fungible Token}-a. Nastavak rada na {\it CryptoKitties} sustavu bi svakako podrazumijevao razvoj korisničkog sučelja, primarno web aplikacije kroz koju bi korisnici mogli koristiti sustav.

\bibliography{literatura}
\bibliographystyle{fer}

\end{document}
