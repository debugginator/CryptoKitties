\documentclass[times, utf8, tehnicka_dokumentacija]{fer}
\usepackage{booktabs}
\usepackage{fancyhdr}
\usepackage{lastpage}
\newarray\tim

\def \naslov {CryptoKitties na Ethereum mreži}
\def \ver {1.0}
\def \broj {4}
\readarray{tim}{Ivan Mihaljević&Blaž Bagić&Karlo Vrbić&Mateo Gobin}
\def \mentor {Federico Matteo Benčić}

\fancypagestyle{plain}

\begin{document}

\title{\naslov}
\verzija{\ver}
\author{\tim}
\size{\broj}
\voditelj{\mentor}

\maketitle
\tableofcontents

\chapter{Uvod}
CryptoKitties je sustav za implementirati sustav uzgajanje i preprodaju virtualnih mačaka. U CryptoKitties sustavu, korisnici prikupljaju i uzgajaju virtualne mačke! Svaka mačka ima jedinstveni genom koji ju definira. Korisnici mogu uzgajati svoje mačke kako bi stvorili nove mačke koje se raylikuju od ostalih po svom genomu.

CryptoKitties je implementiran kao raspodijeljeni sustav zasnovan na {\it blockhain} tehnologiji. {\it Blockhain} je tehnologija pomoću koje implementirati raspodijeljene i decentralizirane sustave, a funkcionira kao lista zapisa, koji nazivamo {\it blokovima}, a koji su povezani jedni s drugima tako tvoreći jedan lanac blokova. Iako CryptoKitties nije digitalna valuta, nudi istu sigurnost: svaka je mačka jedinstvena i zna se točno njen vlasnik. Ne može se ponoviti, oduzeti ili uništiti.

\chapter{Tehnologija}
Sustav je implementiran pomoću {\it Ethereum} blockchain tehnologije. Ethereum je decentralizirana platforma koja ima mogućnost pokretanja tzv. {\it smart contract-a}, tj. aplikacija koje rade točno onako kako su programirane bez ikakve mogućnosti zastoja, cenzure, prijevare ili uplitanja trećih strana. Na taj način možemo grantirati sigurnost pri preprodaji i parenju mačaka.

Za razvoj i testiranje pametnih ugovora korišten je alat {\it Truffle} koji omogućava simuliranje {\it Ethereum} mreže lokalno na osobnom računalu. Ovaj alat nam omogućava kompilaciju pametnih ugovora, automatsko testiranje pomoću {\it JavaScript} biblioteka {\it Mocha} i {\it Chai}, skriptiran deployment i migracije i još mnogo funkcionalnosti koje olakšavaju razvoj pametnih ugovora na {\it Ethereum} mreži.

\chapter{Zaključak}
Zaključak.

\bibliography{literatura}
\bibliographystyle{fer}

\end{document}
